\documentclass[12pt,a4paper,twoside,openany]{book}
\usepackage{pethesis}

% 図が多くビルドに時間がかかるときは
% \documentclass[12pt,a4paper,twoside,openany,draft]{book}
% のように\documentarticleのオプションに「draft」を追加する

% ***** フォントについて *****
% フォントに関してはIPAexフォントとGoogle Notoフォントを
% オプションとして用意していますので、好みの方をお使いください (初期値はIPAフォント)。
% なお斜体のフォントはIPAフォントのみ対応
% \usepackage[jpfont=ipa]{pethesis}
% https://ja.osdn.net/projects/luatex-ja/wiki/LuaTeX-ja%E3%81%AE%E4%BD%BF%E3%81%84%E6%96%B9

% ***** 節の自動改行 *****
% 各節を自動で改ページする場合はautobreaksecオプションを追加
% \usepackage[autobreaksec]{pethesis}

% ***** 追加のパッケージ *****
% !注意!
% グラフィクス用のpackageは既にusepackageしているので、
% \usepackage[dvips]{graphics}などと書くとエラーになる
% (\usepackage[xetex]{graphicx}を使わないといけない)
\usepackage{amsmath,amssymb}
\usepackage{mathtools}
\usepackage{url}
\usepackage{array}
\usepackage{booktabs}
\usepackage{here}
\usepackage{algorithm,algpseudocode}
\usepackage{tabularray}
\UseTblrLibrary{booktabs}

% ***** BibTeXの設定 *****
\usepackage[
  backend=biber,
  backref=true,
  sortcites=true,
  style=ieee,
  citestyle=numeric,
  maxbibnames=999,
  mincitenames=999,
  maxcitenames=999,
  sorting=none,
  url=false,
  doi=true,
  isbn=true
]{biblatex}

% backrefの表記を編集
\DefineBibliographyStrings{english}{%
  backrefpage = {cited on page},    % originally "cited on page"
  backrefpages = {cited on pages},  % originally "cited on pages"
}

% 余計な項目を非表示にしておく (必要な場合はコメントアウトする)
\AtEveryBibitem{
  \clearfield{eprint}%
  \clearfield{date}%
  \clearfield{month}%
  \clearfield{issn}%
  \clearfield{series}%
  \clearlist{address}%

  % "book"以外から削除する項目
  \ifentrytype{book}{}{
    \clearfield{isbn}%
    \clearname{editor}%
    \clearlist{publisher}%
  }
}

% BibTeXのソースを指定
\addbibresource{pethesis.bib}

% ***** リンクの設定 *****
% たまにコンパイルに失敗する可能性あり
% 必ず最後にusepackageすること
\ifxetex
  \usepackage[xetex]{hyperref}
\else
  \ifluatex
    \usepackage[luatex]{hyperref}
  \else
    \usepackage{hyperref}
  \fi
\fi

\hypersetup{
  backref=true,
  bookmarks=true,
  colorlinks=true,
  linktocpage=true,         % 目次においてページ番号にリンクをつける
  linkcolor=[rgb]{0, 0, 1}, % 主に章や節の番号のリンク
  citecolor=[rgb]{0, 0, 1}, % 文献の引用
  urlcolor=[rgb]{0, 0, 1}   % URL
}

% なおリンクを残しつつ、リンクの文字が目立たないようにしたい場合には
% 以下のコマンドに置き換える。
% \hypersetup{hidelinks}

% ***** CleverRefの設定 *****
\usepackage{cleveref}

\newcommand{\crefformatJP}[3]{%
  \crefformat{#1}{#2\,##2##1##3\,#3}%
  \crefrangeformat{#1}{#2\,##3##1##4\,#3から#2\,##5##2##6\,#3}%
  \crefmultiformat{#1}{#2\,##2##1##3\,#3}{ならびに#2\,##2##1##3\,#3}{、#2\,##2##1##3\,#3}{ならびに#2\,##2##1##3\,#3}%
}
\crefformatJP{equation}{式}{}
\crefformatJP{figure}{図}{}
\crefformatJP{table}{表}{}
\crefformatJP{algorithm}{アルゴリズム}{}
\crefformatJP{chapter}{第}{章}
\crefformatJP{section}{第}{節}
\crefformatJP{subsection}{第}{項}
\crefformatJP{appendix}{付録}{}

\renewcommand{\thesection}{\arabic{chapter}.\,\arabic{section}}
\renewcommand{\thesubsection}{\arabic{chapter}.\,\arabic{section}.\,\arabic{subsection}}

\newcommand{\crefpairconjunction}{ならびに}
\newcommand{\crefrangeconjunction}{から}
\newcommand{\crefmiddleconjunction}{、}
\newcommand{\creflastconjunction}{、ならびに}

\theoremstyle{definition}
\newtheorem{theorem}{定理}
\crefformatJP{theorem}{定理}{}
\newtheorem{lemma}{補題}
\crefformatJP{lemma}{補題}{}
\newtheorem*{proof*}{証明}
\crefformatJP{proof*}{証明}{}
\newtheorem{definition}{定義}
\crefformatJP{definition}{定義}{}
\newtheorem{proposition}{命題}
\crefformatJP{proposition}{命題}{}

% ***** 付録中の表示を設定 *****
\AtBeginEnvironment{appendices}{
  \pagestyle{plain}%
  \renewcommand{\thesection}{\Alph{chapter}.\,\arabic{section}}
  \renewcommand{\thesubsection}{\Alph{chapter}.\,\arabic{section}.\,\arabic{subsubsection}}
  \appendix%
}

% ***** 強調表示用のマクロ *****
\newcommand{\BOLD}[1]{{\bfseries\sffamily #1}}
\newcommand{\RED}[1]{\textcolor[rgb]{0.85, 0.10, 0.10}{#1}}

% ***** argmin, argmax *****
\DeclareMathOperator*{\argmax}{argmax}
\DeclareMathOperator*{\argmin}{argmin}

% ***** アルゴリズムの設定 *****
% RequireとEnsureをInputとOutputにする
\renewcommand{\algorithmicrequire}{\textbf{Input:}}
\renewcommand{\algorithmicensure}{\textbf{Output:}}

% ***** 画像を入れるフォルダの設定 *****
\graphicspath{{figs}}

% ***** タイトルページ *****
\thesistype{令和三年度 卒業論文}
\title{とってもとってもとってもとっても\\とってもとってもとってもとっても\\良い卒業論文の書き方}
\etitle{How to write a super super super\\super super super super super\\great bachelor dissertation}
\adjustspace{100truept} %%英文タイトルと指導教員名の間の垂直余白長さの調整。小さい値であるほど余白が狭くなる。タイトルが複数行にわたる場合、値を小さくして調整  
\affiliation{東京大学工学部精密工学科}
\supervisor{鈴木~宏正~教授}
\studentid{99-999999}
\author{泥濘~三角}

% ------------------------------------------------------------------------------
% 実際の文章の開始
% ------------------------------------------------------------------------------
\begin{document}

% ***** 表紙を出力 *****
\frontmatter
\maketitle

% ***** 論文全体の概要 *****
\begin{abstract}
  論文全体の概要を書く。概要は、論文全体の内容がイメージできるものである必要があるので、提案法についてだけでなく、研究背景や、提案法に至る着想、提案法の中身、実験結果、展望も含め全体について簡潔に書くことを心がける。その際、多くのことを主張することは避けて、論文の最も重要な部分、例えば、研究上の新規性であったり、提案法により得られた有効な結果について重点的に述べる。時折、やたら研究背景が長く書いてある概要を見かけるが、最低でも自分が研究したことが半分以上は概要に入るようにする。
\end{abstract}

% ***** 目次たち *****
\tableofcontents  % 目次
\clearpage

\listoffigures    % 図目次
\clearpage

\listoftables     % 表目次
\clearpage

% ------------------------------------------------------------------------------
% 本文の開始
% ------------------------------------------------------------------------------
\mainmatter

% 章ごとにファイルを分割すると良い
% ルートのディレクトリが散らからないようにサブフォルダに入れよう
\chapter{\latex の基本}
\label{chap:latex-basic}

\begin{chapabst}
  章の概要を別にして書きたい場合には「\texttt{chapabst}」コマンド (pethesis.styで定義) を使う。\texttt{chapabst}コマンドを使わない場合、章の概要は表示されない。
\end{chapabst}

\latex を解説している文章は世の中にたくさんあると思われるが、比較的まとまっているのは「Wikibook」にある「LaTeX」の項目だろう。英語だが、平易に書かれており、\latex のソースコードも多く示されているので、読むのに苦労することは少ないと思われる (一応、日本語版もあるが、やや量が少ない)。\latex の基本的な書き方で分からないことがあれば、まずはここを参照すること。
\begin{itemize}
  \item \textsf{Wikibook: LaTeX}~~(\url{https://en.wikibooks.org/wiki/LaTeX})
\end{itemize}

この章では、\cref{sec:write-equation}で式の書き方を、\cref{sec:insert-figure}では図の入れ方を、\cref{sec:insert-table}では表の入れ方を、\cref{sec:write-algorithm}ではアルゴリズムの書き方を、最後の\cref{sec:cite-articles}では文献引用の方法について述べる。

% ------------------------------------------------------------------------------
\section{式の書き方}
\label{sec:write-equation}
% ------------------------------------------------------------------------------

式には主に文中に式を代入するやり方と、横いっぱいの領域に式を入れる方法の二つがあり、前者の方法を特にインラインと呼ぶこともある。例えば、二次方程式を解く場合、多項式の係数を$a, b, c$とすると、一般的な二次方程式は以下のように書ける。
%
\begin{equation}
  a x^2 + b x^2 + c = 0.
  \label{eq:quadratic-equation}
\end{equation}
%
式を書いた後に空白行をいれると、段落が変わったものと見なされてしまうので、通常はこのように\texttt{\textbackslash end\{equation\}}の後に続けて文章を書くか、「\%」を間の空行に入れることで改段落を防げる。また、論文において、式は文章の一部とみなされるため、一般的には、式の最後にピリオドあるいはコンマをつける。

式を引用する際は\texttt{cleveref}の機能を使って\texttt{\textbackslash cref\{eq:quadratic-equation\}}のように書く。出力は\cref{eq:quadratic-equation}のようになる。\texttt{\textbackslash cref}は式だけでなく図や表にも同じように使えて、自動的に図、表と言ったラベルが切り替わる。また、複数の式を一度に引用する場合には\texttt{\textbackslash cref\{eq:quadratic-equation,eq:with-align\}}のようにコンマで区切ると、「\cref{eq:quadratic-equation,eq:with-align,eq:with-gather}」のように、自動的にそれらしい表記に変換してくれる (コンマとラベルの間にスペースは入れられない)。また連続した番号を物色を引用すると、「\cref{eq:example-line-1,eq:example-line-2,eq:with-align,eq:with-gather}」のように表記される。

数式を複数行で表示する時、かつては\texttt{eqnarray}を使うことが多かったが、最近は\texttt{align}あるいは\texttt{gather}を使うのが一般的。\texttt{align}の場合、\texttt{eqnarray}と同様、式を揃える時には\&を使う。
%
\begin{align}
  a &= a', & b &= b', & c &= c', \label{eq:example-line-1} \\
  a &= a', & b &= b', & c &= c'. \label{eq:example-line-2}
\end{align}
%
一方で\texttt{gather}は式を\&の位置で揃えるのではなく、各行を中央ぞろえにする。
%
\begin{titlebox}{\texttt{align}を使った場合}
  \adjustdisplayskips{1mm}
  \begin{align}
    \mathbf{x} = \argmin_{\mathbf{x}} \| \mathbf{Ax} - \mathbf{b} \|^2 \nonumber \\
    \mbox{where} \quad \mathbf{A} \in \mathbb{R}^{n \times m}, \quad \mathbf{b} \in \mathbb{R}^n \label{eq:with-align}
  \end{align}
\end{titlebox}
%
\begin{titlebox}{\texttt{gather}を使った場合}
  \adjustdisplayskips{1mm}
  \begin{gather}
    \mathbf{x} = \argmin_{\mathbf{x}} \| \mathbf{Ax} - \mathbf{b} \|^2 \nonumber \\
    \mbox{where} \quad \mathbf{A} \in \mathbb{R}^{n \times m}, \quad \mathbf{b} \in \mathbb{R}^n \label{eq:with-gather}
  \end{gather}
\end{titlebox}
%
上記の二つの式で\cref{eq:with-align}が\texttt{align}を使ったもの、\cref{eq:with-gather}が\texttt{gather}を使ったものである。

なお、ベクトルや行列の表記についてだが、計算機科学の分野では「Numerical Recipes in C」~\cite{numrecipes}という本の書き方に倣い、ベクトルを$\mathbf{x}$のようなアルファベット小文字のボールド体、行列を$\mathbf{A}$のようなアルファベット大文字のボールド体で書く。応用数学の分野などでは斜体かつボールドでない文字を使う文化もあるが、$\bm{x}$のようなイタリック・ボールド体はあまり使わないので注意。

\subsection{数式の例いろいろ}

\begin{titlebox}{積分}
  \adjustdisplayskips{1mm}
  \begin{equation}
    I = \int_0^{\infty} x^2 \mathrm{d}x
  \end{equation}
  \begin{center}
    ※ $dx$は正式には$\mathrm{d}x$のように$d$を斜体にせずに書く
  \end{center}
\end{titlebox}

\begin{titlebox}{和の計算}
  \adjustdisplayskips{1mm}
  \begin{equation}
    S = \sum_{i=0}^N x^2
  \end{equation}
\end{titlebox}

\begin{titlebox}{行列}
  \adjustdisplayskips{1mm}
  \begin{equation}
    \mathbf{A} = \begin{bmatrix}
      a_{0,0} & a_{0,1} & \cdots & a_{0,n-1} \\
      a_{1,0} & a_{1,1} & \cdots & a_{1,n-1} \\
      \vdots & \vdots & \ddots & \vdots \\
      a_{n-1,0} & a_{n-1,1} & \cdots & a_{n-1,n-1}
    \end{bmatrix}
  \end{equation}
\end{titlebox}

% ------------------------------------------------------------------------------
\section{数値や単位の書き方}
\label{sec:numbers-and-units}
% ------------------------------------------------------------------------------

数値や単位を書く場合は、通常通り数字を書けば問題ないが、\texttt{siunitx}パッケージを使うと便利な場合がある。例えば、有効数字を明らかにして$1.0 \times 10^3$のように書く場合、数式モードで書き下しても良いが、\texttt{siunitx}を用いると\texttt{\textbackslash num\{1.0e3\}}のように書くだけで「\num{1.0e3}」と出力される。またコンマで3桁ごとに区切られた数字も\texttt{siunitx}を用いることで\num{1000000} (\textbackslash num\{1000000\})のように書くことができる (\texttt{pethesis.sty}ファイルにて\texttt{\textbackslash sisetup\{group-separator=\{,\}\}}を指定している)。

単位を付ける場合には、チルダ記号「$\sim$」を使って数字と単位の間に半角スペースを入れ、10~kVのように書く。このような単位付きの表記の場合も、同様に\texttt{siunitx}パッケージを使うと便利な場合がある。同じ出力は\texttt{\textbackslash qty\{10\}\{\textbackslash kV\}}のように書くと得られ、出力は「\qty{10}{\kV}」のようになる。特に$\mathrm{\mu m}$のようにギリシャ文字が含まれるような単位は\texttt{\textbackslash qty\{10\}\{\textbackslash um\}} (出力: \qty{10}{\um})のように書く方がフォントを揃えるのが容易である。この他にも様々な単位が事前定義されているので、詳しくはドキュメントを確認すると良い。なお、角度の記号などはスペースを入れずに書くのが通例で、\texttt{siunitx}では\texttt{\textbackslash ang\{90.0\}}のように書くことで、「\ang{90.0}」という出力が得られる。

% ------------------------------------------------------------------------------
\section{図の入れ方}
\label{sec:insert-figure}
% ------------------------------------------------------------------------------

一昔前は日本語を\latex で使うために「pLaTeX」+「dvipdfmx」を使う必要があり、一度中間のDVIファイルを作る都合から画像ファイルをEPS形式で要する必要があった。 一方、現在は\pdflatex のほか\lualatex や\xelatex など、DVIを介せずに直接PDFを出力できる\tex を使用するため、画像はPDFファイルとして用意するのが一般的である。JPEGファイルを直接取り込むことも可能であるが、できる限りPDFで用意することを推奨する。

画像の取り込みには\texttt{figure}環境と\texttt{includegraphics}命令を以下のように組み合わせる。結果は\cref{fig:remeshing}のようになる。なお、本テンプレートは画像のPDFファイルを\texttt{figs}フォルダに配置しているが\texttt{pethesis.tex}のプリアンブルに\texttt{graphicspath}を指定しているため、このフォルダの中に画像ファイルが入っている場合には、フォルダを別段指定する必要はない。

\begin{lstlisting}[language=tex,caption={図の貼り方}]
  \begin{figure}[tb]
    \centering
    \includegraphics[width=\linewidth]{triangulation}
    \caption{Stanford bunnyモデルのリメッシュ結果}
    \label{fig:remeshing}
  \end{figure}  
\end{lstlisting}

\begin{figure}[!h]
  \centering
  \includegraphics[width=\linewidth]{triangulation}
  \caption{Stanford bunnyモデルのリメッシュ結果}
  \label{fig:remeshing}
\end{figure}

\begin{figure}[tbp]
	\centering
	\begin{tabular}{ccc}
    \includegraphics[width=.3\linewidth]{gtekt1} &
    \includegraphics[width=.3\linewidth]{gtekt2} &
    \includegraphics[width=.3\linewidth]{gtekt3}
  \end{tabular}
	\caption[溶接工程における大型部品運搬の様子]{溶接工程における大型部品運搬の様子 (図は株式会社ジーテクトのウェブページ~\cite{GTekt}より引用)}
	\label{fig:welding}
\end{figure}

また、複数の図を\cref{fig:welding}のような形で、\texttt{figure}と\texttt{tabular}を使ってレイアウトすることもできるが、図のレイアウトの自由度が低く、図の間のスペースが広くなる傾向にあるので、個人的にはおすすめはしない。なるべく、一枚の図を作って取り込む方が見た目はきれいになる。

図の配置については、通常、図はページの上に揃えて配置することが一般的であるので\texttt{\textbackslash \{figure\}[t]}のような形でtopに揃えることを意味するtを添えることが多い。この他、図が下に揃ってもいい場合には「b」、図を1ページを丸々使って表示したい場合には「p」を指定する。その場に表示したい場合には「h」あるいは「!h」を指定すればよいが、通常あまり好ましくないとされている。

キャプションを書く際、特に図目次がある学位論文においては、キャプションが長いせいで図目次が見づらくなることがある。この場合は、\texttt{\textbackslash caption[図目次用のキャプション]\{図の下に出るキャプション\}}のように書くことで、図目次専用のキャプションを指定できる。

% -----------------------------------------------
\subsection{図を入れる際の注意事項}

\begin{figure}[tbp]
  \centering
  \includegraphics[width=\linewidth]{sinogram_polygonizer}
  \caption[Sinogram Polygonizer法の概念図]{Sinogram Polygonizer法の概念図 (Yamanaka et al.~\cite{yamanaka2013sinogram}, \copyright 2013 IEEE)。}
  \label{fig:sinogram-polygonizer}
\end{figure}

図を別の論文から直接貼り付けることは一般的には好ましくなく、学会や論文誌に投稿する論文においては、ほとんどなされることはない。ただし、書籍や学位論文においては、一定のルールの元で引用することは可能と考えられている。厳密には、画像の著作権を有する学会や個人に許可をとった上で、掲載料を払って掲載する。主要な学会ではIEEEなどは学位論文に関しては掲載料が無料だが、ACMはかなり高額な掲載料がかかるので、特に一般に閲覧可能となる博士論文などは正規の手続きが求められる。引用に際しては\cref{fig:sinogram-polygonizer}のように、図のキャプションに論文のコピーライトを掲載する。


% ------------------------------------------------------------------------------
\section{表の書き方}
\label{sec:insert-table}
% ------------------------------------------------------------------------------

表の書き方については詳しく触れないが、テーブルの間の罫線については\texttt{\textbackslash hline}ではなく、\texttt{booktabs}パッケージの\texttt{\textbackslash toprule} (テーブルの上端の線)、\texttt{\textbackslash midrule} (テーブルの中間に使う線)、\texttt{\textbackslash bottomrule}のように太さが異なる線を使う方が見やすい。特に近年のCG分野の研究では、縦線を使うことはほとんどなく、\texttt{\textbackslash cmidrule} を使ってテーブルの横線に切れ目を入れることで区別をする場合が多いように感じる。

加えてテーブルの横幅だが、最近は\cref{tab:num-fruits-tblr}のようにページの幅いっぱいにテーブルを広げているものを多く見る。テーブルを横方向一杯に広げる方法はいくつかあるが、最も簡単なのは\texttt{tabularray}パッケージを使う方法である。\texttt{tabularray}パッケージでは、「\texttt{\textbackslash begin{tblr}\{colspec=\{列フォーマット指定\}, width=\textbackslash linewidth\}}」のように書くことでテーブルの幅を指定できる他、列のフォーマット指定時に「\texttt{lcr}」などの代わりに\texttt{X[l]}, \texttt{X[c]}, \texttt{X[r]}のように書くことで、テーブルの幅を保ったまま、各列の文字揃えを指定できる。これ以外にも、\texttt{{X[2,l]X[1,c]X[1,c]}}のように指定することで、各列の幅の比 (この例では2:1:1)などを指定することもできる。ただし、\texttt{tabularray}パッケージは\latex 3でないと使えないので注意。

\begin{table}[tb]
  \centering
  \caption{\texttt{tblr}環境を使った場合。}
  \label{tab:num-fruits-tblr}
  \begin{tblr}{colspec={X[2,r]X[3,c]X[3,c]X[3,c]}, width=\linewidth}
    \toprule
    名前 & りんご & みかん & バナナ \\
    \cmidrule{1-1} \cmidrule[l]{2-4}  % tabularrayの場合は()が[]に変わるので注意
    個数 & 1個 & 2個 & 3個 \\
    \bottomrule
  \end{tblr}
\end{table}

\begin{table}[tb]
  \centering
  \caption{列の幅を個別に指定した場合。}
  \label{tab:num-fruits-tabular}
  \begin{tblr}{
    column{1}={0.15\linewidth,r},
    column{2,3,4}={0.22\linewidth,c}
  }
    \toprule
    名前~~ & りんご & みかん & バナナ \\
    \cmidrule[r]{1-1} \cmidrule[l]{2-4}
    個数~~ & 1個 & 2個 & 3個 \\
    \bottomrule
  \end{tblr}
\end{table}


% ------------------------------------------------------------------------------
\section{アルゴリズムの書き方}
\label{sec:write-algorithm}
% ------------------------------------------------------------------------------

アルゴリズムの書き方については、機能が多岐に渡るので、冒頭でも紹介したWikibookなどを参照すること。実例として挿入ソートのアルゴリズムを\cref{alg:example-algo}に示してある。
\begin{itemize}
  \item \textsf{LaTeX/Algorithms - Wikibook} \\(\url{https://en.wikibooks.org/wiki/LaTeX/Algorithms})
\end{itemize}

なお、何らかのアルゴリズムを説明する際に、以下のような\texttt{algorithm}環境を用いて擬似コードを記述することは必須ではない。あくまで疑似コードは「本文を補助するためのもの」であり、擬似コードを示しただけでは説明したことにはならない。逆に本文で明快な説明ができているのであれば、あえて擬似コードを掲載する必要はない。

\begin{algorithm}[!h]
  \caption{アルゴリズムの例 (挿入ソート)}
  \label{alg:example-algo}
  \begin{algorithmic}[1]
    \Require{$A$: A list of numbers.}
    \Ensure{$A'$: The sorted list of $A$.}
    \Procedure{Insertion-Sort}{$A$}
        \For{$j = 2 \, \ldots \, A.\mathrm{length}$}
        \State{$k = A[j]$}
            \State{$i = j - 1$}
            \While{$i > 0$ and $A[i] > k$}
                \State{$A[i + 1] = A[i]$}
                \State{$i = i - 1$}
            \EndWhile
            \State{$A[i + 1] = k$}
        \EndFor
    \EndProcedure
  \end{algorithmic}
\end{algorithm}

% ------------------------------------------------------------------------------
\section{文献の引用の仕方}
\label{sec:cite-articles}
% ------------------------------------------------------------------------------

文献の引用と言えば\bibtex を使うのが一般的だったが、最近は機能を拡張した\biblatex というパッケージがあり、この文章では後者を使用している。ほとんど使い方は変わらないが、文献リストをプロパティで細かく設定できる分、\biblatex のほうが自由度が高い。例えばbackrefと呼ばれる文献が何ページで使われているかという情報を文献リストに付け加えることができる。引用の仕方や\texttt{*.bib}ファイルの書き方に関しては、\bibtex と\biblatex で大きな違いはない。ただし、\biblatex を使うためには\bibtex とは異なるスタイル指定のファイルが必要になるので、学会等に論文を投稿する際、\biblatex 用のスタイルファイルが用意されていなければ使うことはできない (\bibtex 用のスタイルファイルはほとんどの場合、提供されている)。

文献を個別に引用する場合には\texttt{\textbackslash cite\{hoppe96progressive\}}のように書く~\cite{hoppe96progressive}。なお、通常引用部分と直前の文章の間にはチルダの記号「$\sim$」を使って半角スペースを入れる。二つ以上を一度に引用する場合には\texttt{\textbackslash cite}の中括弧の中にコンマ区切りで並べるだけでよい~\cite{garland97surface,kazhdan06poisson, hoppe96progressive,qi2017pointnetpp}。なお、この文書では\biblatex の設定で、適当な順で\texttt{cite}の中に並べても番号順にソートされるようになっている。

また、著者名が日本語の場合、そのままだと漢字の最初の文字だけが抜き出されてしまうため、\texttt{author=\{\{谷田川 達也 and 大竹 豊 and 鈴木 宏正\}\}}のように\texttt{*.bib}ファイル内の著者名の部分で波括弧を二重にして囲う必要がある。

\chapter{更新履歴}
\label{chap:change-log}

\begin{chapabst}
  本テンプレートの更新履歴。
\end{chapabst}

\section*{2022年}

\subsection*{2月9日}

\begin{itemize}
  \item 付録内にある節の番号表記を修正
\end{itemize}

\subsection*{1月24日}

\begin{itemize}
  \item 章、図などの引用時のラベルと数字の間のスペースを調整
\end{itemize}

\subsection*{1月22日}

\begin{itemize}
  \item 付録の引用時の表示を修正
\end{itemize}

\subsection*{1月18日}

\begin{itemize}
  \item biberの機能を使ってISBN13に自動でハイフンを挿入
  \item \biblatex の機能を使って、余計な項目 (publisherなど)を自動除去
\end{itemize}

\subsection*{1月17日}

\begin{itemize}
  \item 数式フォントと英字フォントをSTIX2に統一
  \item \texttt{siunitx}の使い方を追加
\end{itemize}

\subsection*{1月13日}

\begin{itemize}
  \item 文章を強調するためにマクロを追加
  \item 論文のセルフチェックリストを追加
\end{itemize}

\subsection*{1月12日}

\begin{itemize}
  \item タイトルが3行以上の場合にスペースが変になる問題を修正
\end{itemize}

\subsection*{1月5日}

\begin{itemize}
  \item \texttt{\textbackslash creflabelformat}の設定を修正
\end{itemize}

\section*{2021年}

\subsection*{11月25日}

\begin{itemize}
  \item 文章の内容を更新
\end{itemize}

\subsection*{11月24日}

\begin{itemize}
  \item \xelatex と\lualatex の両方をサポートするように変更
\end{itemize}

\subsection*{11月18日}

\begin{itemize}
  \item \xelatex から\lualatex に対応エンジンを変更
  \item \texttt{cleveref}を使用するように変更
  \item 句読点を自動的に置換するズボラ設定の追加
  \item 各節の直前に改ページを入れるかどうかのオプションを追加
  \item 表作成のライブラリを\texttt{tabu}から\texttt{tabularray}に変更 (\latex 3が必須)
  \item 箇条書きのマージンを調整
  \item 図表のキャプションを調整
\end{itemize}

\chapter{関連研究}
\label{chap:relwork}

関連研究を書く。関連研究の章で説明する研究の数としては、卒論だと目安20個。修論だと目安30--40個くらい。論文全体で参考文献の数が卒論30以上、修論50以上くらいになるのが目標。

% 卒論などではそうでもないが,一般的な論文の場合には手法を説明する章や節に
% 「提案手法」のような一般名称は使わないことが多い
\chapter{提案手法}
\label{chap:method}

提案手法について書く。既存手法を用いた部分を、今回の論文で新しく提案する部分が明確になるように書くこと。筆者の好みでは、章の名前を提案手法とするのはあまり好ましくなく、具体的に何を提案するのかを表す章タイトルとする方が良い。

\chapter{結果と考察}
\label{chap:results}

ここでは仮に下のような単純な流れにしているが、各実験の意味とつながりを考えて、適切な順番を考える。この時、必ずしも、下の三項目が別々の節になっている必要はない。

\section{実験概要}
\label{sec:expabst}

実験の概要を書く。

\section{実験結果}
\label{sec:results}

実験の結果を書く。

\section{考察}
\label{sec:discussion}

実験を考察する。余談だが、日本語の考察は結果に関する「分析」のような立ち位置を考えられている一方、特に英語論文の場合には「分析」は実験結果も含めて、結果と同じ節にまとめるのが一般的である。逆に、考察では、結果から類推される「可能性」のようなものを議論する。

\chapter{結論と展望}
\label{chap:conclusion}

今回の研究で得られた知見を再度まとめて、今後考えられる発展性や、今回の研究ではやりつくせなかった部分などについて思いを馳せる。

謝辞は、一般的に以下のような順序で書く。順番に迷ったら、オフィシャルなものから始まり、だんだんプライベートなものになるようにする。

\begin{itemize}
  \item 指導教員
  \item 副査の先生
  \item 研究室の指導教員以外の先生
  \item その他 (研究室の先輩・同期・後輩、家族など)
\end{itemize}


% ------------------------------------------------------------------------------
% 参考文献
% ------------------------------------------------------------------------------

% 今回のフォーマットはBibLaTeXを使っているが、普通にBibTeXを使う場合は
% コメントされている部分を変更する

% ***** BibTeX用 *****
% \bibliographystyle{plain}
% \bibliography{pethesis}

% ***** BibLaTeX用 *****
\printbibliography[title=参考文献, heading=bibintoc]

% ------------------------------------------------------------------------------
% 付録
% ------------------------------------------------------------------------------

% 特になければ以下はコメントアウト
\begin{appendices}
  \chapter{一つ目の付録}
\label{chap:first-apdx}

付録には、例えば重要な関連研究で、本文で説明してしまうと長くなりすぎてしまうものや、式の導出課程、本文に乗せると冗長になってしまう実験結果などを書く。ただし、付録が本文に対して長すぎると、若干見苦しいので、付録に載せるにしても、今回の研究と関係の深いものに限るようにする。

\section{一つ目の付録の節}
\label{sec:first-sec-in-first-apdx}

付録を文中で引用すると\cref{chap:first-apdx}や\cref{sec:first-sec-in-first-apdx}のように表示される。

\chapter{二つ目の付録}
\label{chap:second-apdx}

二つ目の付録を書く。

  \chapter{論文セルフチェックリスト}
\label{chap:self-check}

以下のセルフチェックリストは筑波大学の金森由博先生が作成されたチェックリストを元に作成しています。オリジナル版は以下のURLから入手できます。
\begin{itemize}
  \item \url{http://kanamori.cs.tsukuba.ac.jp/docs/writing_paper_checklist.pdf}
\end{itemize}

なお、良い論文を作る上で一番大切なことは「これくらいなら許されるだろう」、「少し面倒だからこれくらいでいいか」と思うところ妥協にせず、\textbf{自分の中で「これ以上はないだろう」と思えるところまで、完成度を高める}ことだと思います。それでも他人が見れば不十分なところはあるものですので、それぐらいの厳しさで望んで初めて及第点のものができるのです。

% {\scriptsize
% \begin{longtblr}
%   [caption={論文セルフチェックリスト}, label={tab:self-check}]
%   {colspec={|X[0.5,c]|X[3,l]|X[5,l]|X[0.5,c]|}, hlines, rows={m}, colsep=1mm}
%     & チェック項目 & 説明 & チェック \\
%   %
%   \multicolumn{4}{l}{\bfseries\sffamily A. 一般} \\
%   A-1 & 章や節の分け方が不適切 & 章、節といったまとまりごとに説明の細かさを揃える & \\
%   A-2 & 章や節のタイトルが不適切 & 章のタイトルが細かすぎたり、節のタイトルが一般的になりすぎたりしていないか & \\
%   A-3 & 章内の節、節内の項などが、1つしかない & 例えば節の中に項が一つしかないのであれば、その項が節になれば良いのであって、一般には不適切。項を増やすか、節にまとめるなどの工夫を施す。 & \\
%   A-4 & 他の論文から文や図をコピペしている & 他の論文から文や図をそのまま持ってくるのは、たとえ自分の論文であったとしても盗用にあたることがある。その場合は、退学、留年、学位剥奪などの厳罰を受けることになるので注意する。ただし、自分の論文をベースにして博士論文をまとめるなどは、許可をとった上で認められる。 & \\
%   A-5 & 各段落がトピックセンテンスではじまっていない & 各段落は「その段落で一番いいたいこと」から書き始める。その上で、トピックセンテンスを徐々に補足していくように修飾文をつけていく。 & \\
%   A-6 & 段落中で接続詞を多用している & 接続詞は便利だが、使いすぎるとメリハリのない文章になって、言いたいことが伝わらなくなる。基本は接続詞がなくても伝わる文章を目指し、特に強調したい箇所に接続詞を使う。 & \\
%   A-7 & 同じ意味の言葉に別の単語が使われている & 論文の中では極力、同じ意味の単語は1種類に統一する。 & \\
%   A-8 & 説明なしに略語や非自明な固有名詞が使われている & 略語を使う場合には、何の略なのかをきちんと書く。また、一般的にそれほど知られていないと思われる固有名詞についても、どのような意味なのかを初出の場所で説明する。 & \\
%   A-9 & 数字と単位の間にスペースが入っていない & 角度などの一部の記号を除き、一般的には数値と単位の間にはスペースが入って、100~mのように表記される。やや面倒だが \texttt{siunitx}パッケージを使って\qty{100}{\meter} (\texttt{\backslash qty\{100\}\{\textbackslash meter\}})のように書くこともできる。 & \\
%   A-10 & 補足説明と思われる内容が括弧書きされている & 補足説明であっても、読んでほしいと思う内容は括弧書きにせずに、本文にそのまま書く。括弧はあくまで略語の説明や、その説明が対応する図を指し示す場合など、重要性が低いもののみに使用する & \\
%   A-11 & 過度な強調後を多用している & 「全く」、「大変」、「非常に」 (英語の場合ならsignificantly, extremelyなど)の強調語は、メリハリがなくなるので、本当に強調したい時以外は使わない。 & \\
%   A-12 & 段落分けが細かすぎる & もちろん基本は言いたいことを一段落にすべきだが、例えば1--2行の段落ができてしまうということは、文章の校正の仕方に問題がある可能性があるので、他の段落とまとめられないかを検討すべき。 & \\
%   %
%   \multicolumn{4}{l}{\bfseries\sffamily B. 図表} \\
%   B-1 & 図、表が文中で引用されていない & 図、表は必ず本文中で引用する。これらはあくまで本文を補助するためのものであるので、図を示して終わり、というのは認められない。 & \\
%   B-2 & 図、表のキャプションが曖昧 & 図表のキャプションはそれが何を表すものなのかが「明確に」分かるように書く。中には図表だけを眺めて論文を読んだことにする人もいるので、少し詳しめに書くほうが望ましい。 & \\
%   B-3 & 図の中に使われている画像の解像度が著しく低い & 理想的には写真以外の図はすべてベクター形式で作成し、EPSやPDFの形式で論文に取り込む。PowerPointなどを使っても作成できる。 & \\
%   B-4 & 図の中の文字が小さい、大きさがバラバラ & 図の中の文字は、最低でも本文に使われているフォントサイズ以上のものを使う。また異なる図であっても、論文全体でフォントサイズがおおよそ同じになるようにサイズを調整する。 & \\
%   B-5 & 図の一部が空白になっている & 図を作るときはなるべく空白ができないように要素を配置する。加えて、図を論文に取り込んだときに、左右に空白が出るような図は極力作らない。 & \\
%   B-6 & 図の見栄えが悪い & 一般的な話として、図に使われている色が原色に近かったり、色が使われすぎたりすると、チープな印象になる。色使いなどについては「ノンデザイナーズ・デザインブック」~\cite{nondesigners}などの書籍が参考になる。 & \\
%   B-7 & 図の中で矢印などの指示記号を使いすぎている & 粗悪な図によく見られるのがレイアウトが適当で矢印で無理やり流れをつくっているもの。レイアウトには左から右、上から下、などの一定の順序を表すルールがあるので、それに従う。 & \\
%   B-8 & ソフトウェアのスクリーンショットをそのまま図に使っている & ソフトウェアのUIなどの不要な情報が含まれて、完成度が低い印象を与えるので、必ず画像部分だけを切り出して、自分でレイアウトする。  & \\
%   B-9 & 図中の結果画像の切り出し範囲が揃っていない & 図に複数のデータに関する結果 (画像やメッシュ)を載せるときには、それぞれの見ている範囲が「完璧に」揃うようにする。例えば、画像の拡大範囲が少しでもずれていると、結果として不適切なだけでなく、完成度が低い印象を与える。 & \\
%   B-10 & グラフの縦軸、横軸が何を表すかが書かれていない & グラフの軸は、縦軸、横軸が何を表すかに加えて、単位があれば、必ず単位も明記する。 & \\
%   B-11 & 文中で「図中の○○が...」と説明している箇所がどこなのか分からない & 図の一部について詳しい説明を入れたい場合は、該当部分を丸で囲うなどして、ひと目でその場所がどこだか分かるようにする。その際、図の重要な部分が隠れないように注意する & \\
%   B-12 & 図と本文中での参照が極端に離れている & 特別な理由 (図が多くどうしても離れてしまう)などの理由がない場合、図と本文中での引用箇所 (最初に出てくる場所)は同じページになるように調整する (表についても同様)。 & \\
%   %
%   \multicolumn{4}{l}{\bfseries\sffamily C. 数式} \\
%   C-1 & 数式の中の記号が説明されていない & 数式の中の記号は、例えば絶対値の記号 $| \cdot |$のような、あまりに一般的なものでない限りは、文字が何を表すかも含めて全て説明する。ただし、各記号を説明するときは、適宜に説明文の中に入れ込むなど、「○○は○○を表す」のような冗長な文章が延々と続くことがないように気をつける & \\
%   C-2 & 変数がスカラー、ベクトル、行列のどれか分からない & 応用数学に近い分野を除いて、計算機科学分野ではスカラーは $a$ (細字イタリック小文字)、ベクトルは$\mathbf{x}$ (太字ブロック小文字)、行列は$\mathbf{A}$ (太字ブロック大文字)で表すことが多い。その他にも集合は $\mathcal{S}$ (\texttt{\backslash mathcal\{S\}}) で表記されるカリグラフィ文字で表記されることが多いなど、一定のルールがある。 & \\
%   C-3 & 数式の直後の文章が字下げされている & 数式は段落の一部、という位置づけなので、数式の後で意図して段落を変えたい時以外は\texttt{\backslash \{equation\}}の後、改行を入れずに文章を続けて書く。あるいは改行の先頭にコメントアウトを表す「\%」を入れても良い。 & \\
%   C-4 & 本文中に数式モードの文字と平文の文字が混同している & 数式を書く場合には、たとえ本文中であっても、もれなく\$ $\cdots$ \$で囲んで数式モードにする。そうしないと、数式ごとにフォントが変わってしまって、完成度が低い印象を与える。 & \\
%   C-5 & 数式内の式展開が細かすぎる & 大学入学当時に数学の教科書を読んで「説明が少ない」と思ったことを思い出してほしい。単に説明が多ければ良いというわけではなく、読者が要点に集中できるように、どこまでを説明するかを決めることが重要。 & \\
%   %
%   \multicolumn{4}{l}{\bfseries\sffamily D. 論文タイトル} \\
%   D-1 & 論文タイトルから提案法の特徴が分からない & 提案法の新規性や従来法との差分が分かるようなタイトルにする & \\
%   D-2 & タイトルが長過ぎる、短すぎる & タイトルに過度に詳細な情報を入れたり、逆に技術貢献に合わない一般的すぎるタイトルをつけている & \\
%   %
%   \multicolumn{4}{l}{\bfseries\sffamily E. 著者名} \\
%   E-1 & 著者の名前、所属、住所などが間違っている & 名前の漢字やスペル、所属は学部、学科が入っているかなどをきちんと確認する & \\
%   %
%   \multicolumn{4}{l}{\bfseries\sffamily F. 論文概要} \\
%   F-1 & 論文概要のほとんどが研究背景や関連研究の説明に使われている & 論文は自分の新しい研究や、その結果について述べるものなので、結論を含めた研究の内容が、最低でも半分以上になるように内容を考える & \\
%   F-2 & 概要の時点で図表や参考文献を引用している & 一般に概要はウェブページ等々で論文本体とは別に扱われるので、本文中の図表や、参考文献を引用することは避ける。 & \\
%   %
%   \multicolumn{4}{l}{\bfseries\sffamily G. 序論} \\
%   G-1 & 序論が具体的すぎる話から始まる & 一例として、例えば特定の研究や手法に関する説明から序論が始まるのは、研究の意義が薄い印象を与えることがある。ある程度、社会的な背景などの大きな問題から序論を書き始めることで、提案法がそのような大問題の解決にも寄与することを説明することもなる。ただし、学会論文の場合はページ数制限があるので、時と場合による。 & \\
%   G-2 & 序論を読んでも、研究の貢献や従来法との違いが分からない & 序論は大きな社会的背景から書き始めて、徐々に具体化して、従来法の問題を述べ、最後にその問題の解決という位置づけで提案法の概要を述べる。貢献についてはリストアップするという手もある。 & \\
%   %
%   \multicolumn{4}{l}{\bfseries\sffamily H. 関連研究} \\
%   H-1 & 関連研究が自分の研究との「関連」を説明していない & 関連研究は単なる「過去の研究」の紹介ではなく、それらの研究が自分の研究とどのように「関連」しているかを述べる場所である。従って、過去の研究を説明したあとで、提案法の目的とそれらがどう関係するのかを述べる。 & \\
%   H-2 & 関連研究の列挙に脈略がない & 関連研究が単に「手法Aは○○、手法Bは○○」のように書かれているのは、読み手に不親切。必ず関連研究をいくつかのグループに分けて整理したあとで、グループごとに手法の特徴を述べていく。 & \\
%   H-3 & 関連研究の説明が単なる悪口になっている & おそらく引用している論文は立派な論文誌や学会誌に掲載されているはずなので、悪いところだけということはありえない。完璧でない部分について述べるにしても、どういう部分が優れているのかを一緒に説明する。 & \\
%   %
%   \multicolumn{4}{l}{\bfseries\sffamily I. 提案手法の説明} \\
%   I-1 & 言葉だけでは絶対伝わらない内容を言葉だけで説明している & 提案法を説明するときには、概念図のようなものをいくつか作って、その概念図にそって説明を進める。また数式で書き表したほうが明確になる場合には、多少冗長でも数式を入れる。 & \\
%   I-2 & どこが新しい方法で、どこが従来法をそのまま使っているのかが分からない & 論文を書く上で、自分の新しい提案が何かを明確に読み手に伝えることは最も重要なので、従来法と提案法の境目が明瞭となるような書き方に努める。一案として、従来法を「背景」のような節や項にまとめてしまうことも可能。 &  \\
%   %
%   \multicolumn{4}{l}{\bfseries\sffamily J. 結果と考察} \\
%   J-1 & 雑多な内容が単に羅列されている & 論文は日記ではないので、こうして、こうなった、ということだけを羅列するのは不適切。まず、その結果を使って言いたいことを考えて、その言いたいことが最大限伝わるように結果を並べ、説明する。 & \\
%   J-2 & 実験条件に対してa, b, cのような単純なラベル付けだけがなされている & 特に、実験条件が多い場合などは、その名前からどういう実験条件なのかが分かるように、名前をつけた上で、各実験条件について説明するときは、どういう実験条件なのかを読者に再度説明する & \\
%   %
%   \multicolumn{4}{l}{\bfseries\sffamily K. まとめ} \\
%   K-1 & 今後の展望がいわゆる「小泉進次郎」構文になっている & 今後の展望は非自明にできないことについて、関連研究なども上げつつ、実現の可能性を議論する場所であり、今のところ提案法でできていないことについて、「できてないので今後の課題」と書くのは不適切。 & \\
%   %
%   \multicolumn{4}{l}{\bfseries\sffamily L. 参考文献} \\
%   L-1 & 参考文献が著しく少ない & 現在は、行っている研究が全くなされていない、ということは非常に稀であるため、参考文献を示して、これまでどのような研究がなされてきたかを説明する必要がある。学会論文なら30前後、卒論30前後、修論50前後、博士論文100以上は参考文献があるのが一般的。もちろん分野によっても異なる。 & \\
%   L-2 & 本文で引用されていない参考文献がリストにある & \bibtex をつかっていればあまり起こらないが、本文で引用されていない参考文献がリストにある場合があるので注意して確認する。 & \\
%   L-3 & 文献のタイトルなどで、略語などが小文字になっている & \bibtex の設定によっては、タイトル中の単語を強制的に小文字にするものもあるため、例えばGPUなどの略語やLambertianなどの大文字を含むべき単語が小文字になっていたら、\texttt{*.bib}ファイルを開いて、該当部分を$\{ \cdots \}$で囲む。 & \\
%   L-4 & 論文誌や学会の予稿集に巻号やページの番号などがない & 少なくとも論文誌であれば、必ず何巻 (volume)、何号 (number)、ページ番号が割り当てられているはずなので、それを正しく書く。 & \\
%   L-5 & 項目によって掲載されている情報量が異なる & 例えば一部の引用にはURLやISBNなどが気まぐれにかかれているのに、そうでないものも多くあるなどの場合、一貫性がない印象を与えるので、全て書くか、全て書かないかのいずれかにできる限り揃える & \\
%   L-6 & 論文の著者の名字と名前が逆になっている & 一般的には「名前 → 名字 (T. Yatagawa)」の順にかかれているが、論文誌によってはウェブサイト等で名字が先に書かれていることもある。名字が先の場合には、Yatagawa, T. などのように名字の後にコンマが入るので、そこに注意する他、Google Scholarなどの論文検索サイトで確認をする。 & \\
%   L-7 & 参考文献の引用と直前の単語の前にスペースがない & 日英ともに、論文の引用と直前の文章の間には「山中らの手法~\cite{yamanaka2013sinogram}」のようにスペースを入れるのが一般的。 & \\
%   %
%   \multicolumn{4}{l}{\bfseries\sffamily M. 日本語特有} \\
%   M-1 & 主語が明確でない文章になっている & 日本語は言葉としては主語を省略可能だが、論文に書く文章としてはふさわしくない。各文が正しく、主語と述語を備えているかを確認する。 & \\
%   M-2 & いわゆる話し言葉がそのまま文章になっている & 「なんとなく」、「微妙に」のような話し言葉基調の言葉は使わない。それほど極端でなくとも、例えば「CTの値」(= X線CT装置により得られた再構成ボリューム上の輝度値)のような明らかに説明不足な言葉も避ける。
% \end{longtblr}
% }

\end{appendices}

% ------------------------------------------------------------------------------
% 論文の最後 (絶対消さない)
\end{document}
