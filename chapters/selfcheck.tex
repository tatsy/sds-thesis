\chapter{論文セルフチェックリスト}
\label{chap:self-check}

以下のセルフチェックリストは筑波大学の金森由博先生が作成されたチェックリストを元に作成しています。オリジナル版は以下のURLから入手できます。
\begin{itemize}
  \item \url{http://kanamori.cs.tsukuba.ac.jp/docs/writing_paper_checklist.pdf}
\end{itemize}

なお、良い論文を作る上で一番大切なことは「これくらいなら許されるだろう」、「少し面倒だからこれくらいでいいか」と思うところ妥協にせず、\BOLD{自分の中で「これ以上はないだろう」と思えるところまで、完成度を高める}ことだと思います。それでも他人が見れば不十分なところはあるものですので、それぐらいの厳しさで望んで初めて及第点のものができるのです。

{\scriptsize
\begin{longtblr}
  [caption={論文セルフチェックリスト}, label={tab:self-check}]
  {colspec={|X[0.5,c]|X[3,l]|X[5,l]|X[0.5,c]|}, hlines, rows={m}, colsep=1mm}
    & チェック項目 & 説明 & チェック \\
  %
  \multicolumn{4}{l}{\bfseries\sffamily A. 一般} \\
  A-1 & 章や節の分け方が不適切 & 章、節といったまとまりごとに説明の細かさを揃える & \\
  A-2 & 章や節のタイトルが不適切 & 章のタイトルが細かすぎたり、節のタイトルが一般的になりすぎたりしていないか & \\
  A-3 & 章内の節、節内の項などが、1つしかない & 例えば節の中に項が一つしかないのであれば、その項が節になれば良いのであって、一般には不適切。項を増やすか、節にまとめるなどの工夫を施す。 & \\
  A-4 & 他の論文から文や図をコピペしている & 他の論文から文や図をそのまま持ってくるのは、たとえ自分の論文であったとしても盗用にあたることがある。その場合は、退学、留年、学位剥奪などの厳罰を受けることになるので注意する。ただし、自分の論文をベースにして博士論文をまとめるなどは、許可をとった上で認められる。 & \\
  A-5 & 各段落がトピックセンテンスではじまっていない & 各段落は「その段落で一番いいたいこと」から書き始める。その上で、トピックセンテンスを徐々に補足していくように修飾文をつけていく。 & \\
  A-6 & 段落中で接続詞を多用している & 接続詞は便利だが、使いすぎるとメリハリのない文章になって、言いたいことが伝わらなくなる。基本は接続詞がなくても伝わる文章を目指し、特に強調したい箇所に接続詞を使う。 & \\
  A-7 & 同じ意味の言葉に別の単語が使われている & 論文の中では極力、同じ意味の単語は1種類に統一する。 & \\
  A-8 & 説明なしに略語や非自明な固有名詞が使われている & 略語を使う場合には、何の略なのかをきちんと書く。また、一般的にそれほど知られていないと思われる固有名詞についても、どのような意味なのかを初出の場所で説明する。 & \\
  A-9 & 数字と単位の間にスペースが入っていない & 角度などの一部の記号を除き、一般的には数値と単位の間にはスペースが入って、100~mのように表記される。やや面倒だが \texttt{siunitx}パッケージを使って\qty{100}{\meter} (\texttt{\backslash qty\{100\}\{\textbackslash meter\}})のように書くこともできる。 & \\
  A-10 & 補足説明と思われる内容が括弧書きされている & 補足説明であっても、読んでほしいと思う内容は括弧書きにせずに、本文にそのまま書く。括弧はあくまで略語の説明や、その説明が対応する図を指し示す場合など、重要性が低いもののみに使用する & \\
  A-11 & 過度な強調後を多用している & 「全く」、「大変」、「非常に」 (英語の場合ならsignificantly, extremelyなど)の強調語は、メリハリがなくなるので、本当に強調したい時以外は使わない。 & \\
  A-12 & 段落分けが細かすぎる & もちろん基本は言いたいことを一段落にすべきだが、例えば1--2行の段落ができてしまうということは、文章の校正の仕方に問題がある可能性があるので、他の段落とまとめられないかを検討すべき。 & \\
  %
  \multicolumn{4}{l}{\bfseries\sffamily B. 図表} \\
  B-1 & 図、表が文中で引用されていない & 図、表は必ず本文中で引用する。これらはあくまで本文を補助するためのものであるので、図を示して終わり、というのは認められない。 & \\
  B-2 & 図、表のキャプションが曖昧 & 図表のキャプションはそれが何を表すものなのかが「明確に」分かるように書く。中には図表だけを眺めて論文を読んだことにする人もいるので、少し詳しめに書くほうが望ましい。 & \\
  B-3 & 図の中に使われている画像の解像度が著しく低い & 理想的には写真以外の図はすべてベクター形式で作成し、EPSやPDFの形式で論文に取り込む。PowerPointなどを使っても作成できる。 & \\
  B-4 & 図の中の文字が小さい、大きさがバラバラ & 図の中の文字は、最低でも本文に使われているフォントサイズ以上のものを使う。また異なる図であっても、論文全体でフォントサイズがおおよそ同じになるようにサイズを調整する。 & \\
  B-5 & 図の一部が空白になっている & 図を作るときはなるべく空白ができないように要素を配置する。加えて、図を論文に取り込んだときに、左右に空白が出るような図は極力作らない。 & \\
  B-6 & 図の見栄えが悪い & 一般的な話として、図に使われている色が原色に近かったり、色が使われすぎたりすると、チープな印象になる。色使いなどについては「ノンデザイナーズ・デザインブック」~\cite{nondesigners}などの書籍が参考になる。 & \\
  B-7 & 図の中で矢印などの指示記号を使いすぎている & 粗悪な図によく見られるのがレイアウトが適当で矢印で無理やり流れをつくっているもの。レイアウトには左から右、上から下、などの一定の順序を表すルールがあるので、それに従う。 & \\
  B-8 & ソフトウェアのスクリーンショットをそのまま図に使っている & ソフトウェアのUIなどの不要な情報が含まれて、完成度が低い印象を与えるので、必ず画像部分だけを切り出して、自分でレイアウトする。  & \\
  B-9 & 図中の結果画像の切り出し範囲が揃っていない & 図に複数のデータに関する結果 (画像やメッシュ)を載せるときには、それぞれの見ている範囲が「完璧に」揃うようにする。例えば、画像の拡大範囲が少しでもずれていると、結果として不適切なだけでなく、完成度が低い印象を与える。 & \\
  B-10 & グラフの縦軸、横軸が何を表すかが書かれていない & グラフの軸は、縦軸、横軸が何を表すかに加えて、単位があれば、必ず単位も明記する。 & \\
  B-11 & 文中で「図中の○○が...」と説明している箇所がどこなのか分からない & 図の一部について詳しい説明を入れたい場合は、該当部分を丸で囲うなどして、ひと目でその場所がどこだか分かるようにする。その際、図の重要な部分が隠れないように注意する & \\
  B-12 & 図と本文中での参照が極端に離れている & 特別な理由 (図が多くどうしても離れてしまう)などの理由がない場合、図と本文中での引用箇所 (最初に出てくる場所)は同じページになるように調整する (表についても同様)。 & \\
  %
  \multicolumn{4}{l}{\bfseries\sffamily C. 数式} \\
  C-1 & 数式の中の記号が説明されていない & 数式の中の記号は、例えば絶対値の記号 $| \cdot |$のような、あまりに一般的なものでない限りは、文字が何を表すかも含めて全て説明する。ただし、各記号を説明するときは、適宜に説明文の中に入れ込むなど、「○○は○○を表す」のような冗長な文章が延々と続くことがないように気をつける & \\
  C-2 & 変数がスカラー、ベクトル、行列のどれか分からない & 応用数学に近い分野を除いて、計算機科学分野ではスカラーは $a$ (細字イタリック小文字)、ベクトルは$\mathbf{x}$ (太字ブロック小文字)、行列は$\mathbf{A}$ (太字ブロック大文字)で表すことが多い。その他にも集合は $\mathcal{S}$ (\texttt{\backslash mathcal\{S\}}) で表記されるカリグラフィ文字で表記されることが多いなど、一定のルールがある。 & \\
  C-3 & 数式の直後の文章が字下げされている & 数式は段落の一部、という位置づけなので、数式の後で意図して段落を変えたい時以外は\texttt{\backslash \{equation\}}の後、改行を入れずに文章を続けて書く。あるいは改行の先頭にコメントアウトを表す「\%」を入れても良い。 & \\
  C-4 & 本文中に数式モードの文字と平文の文字が混同している & 数式を書く場合には、たとえ本文中であっても、もれなく\$ $\cdots$ \$で囲んで数式モードにする。そうしないと、数式ごとにフォントが変わってしまって、完成度が低い印象を与える。 & \\
  C-5 & 数式内の式展開が細かすぎる & 大学入学当時に数学の教科書を読んで「説明が少ない」と思ったことを思い出してほしい。単に説明が多ければ良いというわけではなく、読者が要点に集中できるように、どこまでを説明するかを決めることが重要。 & \\
  %
  \multicolumn{4}{l}{\bfseries\sffamily D. 論文タイトル} \\
  D-1 & 論文タイトルから提案法の特徴が分からない & 提案法の新規性や従来法との差分が分かるようなタイトルにする & \\
  D-2 & タイトルが長過ぎる、短すぎる & タイトルに過度に詳細な情報を入れたり、逆に技術貢献に合わない一般的すぎるタイトルをつけている & \\
  %
  \multicolumn{4}{l}{\bfseries\sffamily E. 著者名} \\
  E-1 & 著者の名前、所属、住所などが間違っている & 名前の漢字やスペル、所属は学部、学科が入っているかなどをきちんと確認する & \\
  %
  \multicolumn{4}{l}{\bfseries\sffamily F. 論文概要} \\
  F-1 & 論文概要のほとんどが研究背景や関連研究の説明に使われている & 論文は自分の新しい研究や、その結果について述べるものなので、結論を含めた研究の内容が、最低でも半分以上になるように内容を考える & \\
  F-2 & 概要の時点で図表や参考文献を引用している & 一般に概要はウェブページ等々で論文本体とは別に扱われるので、本文中の図表や、参考文献を引用することは避ける。 & \\
  %
  \multicolumn{4}{l}{\bfseries\sffamily G. 序論} \\
  G-1 & 序論が具体的すぎる話から始まる & 一例として、例えば特定の研究や手法に関する説明から序論が始まるのは、研究の意義が薄い印象を与えることがある。ある程度、社会的な背景などの大きな問題から序論を書き始めることで、提案法がそのような大問題の解決にも寄与することを説明することもなる。ただし、学会論文の場合はページ数制限があるので、時と場合による。 & \\
  G-2 & 序論を読んでも、研究の貢献や従来法との違いが分からない & 序論は大きな社会的背景から書き始めて、徐々に具体化して、従来法の問題を述べ、最後にその問題の解決という位置づけで提案法の概要を述べる。貢献についてはリストアップするという手もある。 & \\
  %
  \multicolumn{4}{l}{\bfseries\sffamily H. 関連研究} \\
  H-1 & 関連研究が自分の研究との「関連」を説明していない & 関連研究は単なる「過去の研究」の紹介ではなく、それらの研究が自分の研究とどのように「関連」しているかを述べる場所である。従って、過去の研究を説明したあとで、提案法の目的とそれらがどう関係するのかを述べる。 & \\
  H-2 & 関連研究の列挙に脈略がない & 関連研究が単に「手法Aは○○、手法Bは○○」のように書かれているのは、読み手に不親切。必ず関連研究をいくつかのグループに分けて整理したあとで、グループごとに手法の特徴を述べていく。 & \\
  H-3 & 関連研究の説明が単なる悪口になっている & おそらく引用している論文は立派な論文誌や学会誌に掲載されているはずなので、悪いところだけということはありえない。完璧でない部分について述べるにしても、どういう部分が優れているのかを一緒に説明する。 & \\
  %
  \multicolumn{4}{l}{\bfseries\sffamily I. 提案手法の説明} \\
  I-1 & 言葉だけでは絶対伝わらない内容を言葉だけで説明している & 提案法を説明するときには、概念図のようなものをいくつか作って、その概念図にそって説明を進める。また数式で書き表したほうが明確になる場合には、多少冗長でも数式を入れる。 & \\
  I-2 & どこが新しい方法で、どこが従来法をそのまま使っているのかが分からない & 論文を書く上で、自分の新しい提案が何かを明確に読み手に伝えることは最も重要なので、従来法と提案法の境目が明瞭となるような書き方に努める。一案として、従来法を「背景」のような節や項にまとめてしまうことも可能。 &  \\
  %
  \multicolumn{4}{l}{\bfseries\sffamily J. 結果と考察} \\
  J-1 & 雑多な内容が単に羅列されている & 論文は日記ではないので、こうして、こうなった、ということだけを羅列するのは不適切。まず、その結果を使って言いたいことを考えて、その言いたいことが最大限伝わるように結果を並べ、説明する。 & \\
  J-2 & 実験条件に対してa, b, cのような単純なラベル付けだけがなされている & 特に、実験条件が多い場合などは、その名前からどういう実験条件なのかが分かるように、名前をつけた上で、各実験条件について説明するときは、どういう実験条件なのかを読者に再度説明する & \\
  %
  \multicolumn{4}{l}{\bfseries\sffamily K. まとめ} \\
  K-1 & 今後の展望がいわゆる「小泉進次郎」構文になっている & 今後の展望は非自明にできないことについて、関連研究なども上げつつ、実現の可能性を議論する場所であり、今のところ提案法でできていないことについて、「できてないので今後の課題」と書くのは不適切。 & \\
  %
  \multicolumn{4}{l}{\bfseries\sffamily L. 参考文献} \\
  L-1 & 参考文献が著しく少ない & 現在は、行っている研究が全くなされていない、ということは非常に稀であるため、参考文献を示して、これまでどのような研究がなされてきたかを説明する必要がある。学会論文なら30前後、卒論30前後、修論50前後、博士論文100以上は参考文献があるのが一般的。もちろん分野によっても異なる。 & \\
  L-2 & 本文で引用されていない参考文献がリストにある & \bibtex をつかっていればあまり起こらないが、本文で引用されていない参考文献がリストにある場合があるので注意して確認する。 & \\
  L-3 & 文献のタイトルなどで、略語などが小文字になっている & \bibtex の設定によっては、タイトル中の単語を強制的に小文字にするものもあるため、例えばGPUなどの略語やLambertianなどの大文字を含むべき単語が小文字になっていたら、\texttt{*.bib}ファイルを開いて、該当部分を$\{ \cdots \}$で囲む。 & \\
  L-4 & 論文誌や学会の予稿集に巻号やページの番号などがない & 少なくとも論文誌であれば、必ず何巻 (volume)、何号 (number)、ページ番号が割り当てられているはずなので、それを正しく書く。 & \\
  L-5 & 項目によって掲載されている情報量が異なる & 例えば一部の引用にはURLやISBNなどが気まぐれにかかれているのに、そうでないものも多くあるなどの場合、一貫性がない印象を与えるので、全て書くか、全て書かないかのいずれかにできる限り揃える & \\
  L-6 & 論文の著者の名字と名前が逆になっている & 一般的には「名前 → 名字 (T. Yatagawa)」の順にかかれているが、論文誌によってはウェブサイト等で名字が先に書かれていることもある。名字が先の場合には、Yatagawa, T. などのように名字の後にコンマが入るので、そこに注意する他、Google Scholarなどの論文検索サイトで確認をする。 & \\
  L-7 & 参考文献の引用と直前の単語の前にスペースがない & 日英ともに、論文の引用と直前の文章の間には「山中らの手法~\cite{yamanaka2013sinogram}」のようにスペースを入れるのが一般的。 & \\
  %
  \multicolumn{4}{l}{\bfseries\sffamily M. 日本語特有} \\
  M-1 & 主語が明確でない文章になっている & 日本語は言葉としては主語を省略可能だが、論文に書く文章としてはふさわしくない。各文が正しく、主語と述語を備えているかを確認する。 & \\
  M-2 & いわゆる話し言葉がそのまま文章になっている & 「なんとなく」、「微妙に」のような話し言葉基調の言葉は使わない。それほど極端でなくとも、例えば「CTの値」(= X線CT装置により得られた再構成ボリューム上の輝度値)のような明らかに説明不足な言葉も避ける。
\end{longtblr}
}
