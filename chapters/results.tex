\chapter{結果と考察}
\label{chap:results}

ここでは仮に下のような単純な流れにしているが、各実験の意味とつながりを考えて、適切な順番を考える。この時、必ずしも、下の三項目が別々の節になっている必要はない。

\section{実験概要}
\label{sec:expabst}

実験の概要を書く。

\section{実験結果}
\label{sec:results}

実験の結果を書く。

\section{考察}
\label{sec:discussion}

実験を考察する。余談だが、日本語の考察は結果に関する「分析」のような立ち位置を考えられている一方、特に英語論文の場合には「分析」は実験結果も含めて、結果と同じ節にまとめるのが一般的である。逆に、考察では、結果から類推される「可能性」のようなものを議論する。
