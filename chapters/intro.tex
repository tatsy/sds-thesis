\chapter{\latex の基本}
\label{chap:latex-basic}

\begin{chapabst}
  章の概要を別にして書きたい場合には「\texttt{chapabst}」コマンド (pethesis.styで定義) を使う。\texttt{chapabst}コマンドを使わない場合、章の概要は表示されない。
\end{chapabst}

\latex を解説している文章は世の中にたくさんあると思われるが、比較的まとまっているのは「Wikibook」にある「LaTeX」の項目だろう。英語だが、平易に書かれており、\latex のソースコードも多く示されているので、読むのに苦労することは少ないと思われる (一応、日本語版もあるが、やや量が少ない)。\latex の基本的な書き方で分からないことがあれば、まずはここを参照すること。
\begin{itemize}
  \item \textsf{Wikibook: LaTeX}~~(\url{https://en.wikibooks.org/wiki/LaTeX})
\end{itemize}

この章では、\cref{sec:write-equation}で式の書き方を、\cref{sec:insert-figure}では図の入れ方を、\cref{sec:insert-table}では表の入れ方を、\cref{sec:write-algorithm}ではアルゴリズムの書き方を、最後の\cref{sec:cite-articles}では文献引用の方法について述べる。

% ------------------------------------------------------------------------------
\section{式の書き方}
\label{sec:write-equation}
% ------------------------------------------------------------------------------

式には主に文中に式を代入するやり方と、横いっぱいの領域に式を入れる方法の二つがあり、前者の方法を特にインラインと呼ぶこともある。例えば、二次方程式を解く場合、多項式の係数を$a, b, c$とすると、一般的な二次方程式は以下のように書ける。
%
\begin{equation}
  a x^2 + b x^2 + c = 0.
  \label{eq:quadratic-equation}
\end{equation}
%
式を書いた後に空白行をいれると、段落が変わったものと見なされてしまうので、通常はこのように\texttt{\textbackslash end\{equation\}}の後に続けて文章を書くか、「\%」を間の空行に入れることで改段落を防げる。また、論文において、式は文章の一部とみなされるため、一般的には、式の最後にピリオドあるいはコンマをつける。

式を引用する際は\texttt{cleveref}の機能を使って\texttt{\textbackslash cref\{eq:quadratic-equation\}}のように書く。出力は\cref{eq:quadratic-equation}のようになる。\texttt{\textbackslash cref}は式だけでなく図や表にも同じように使えて、自動的に図、表と言ったラベルが切り替わる。また、複数の式を一度に引用する場合には\texttt{\textbackslash cref\{eq:quadratic-equation,eq:with-align\}}のようにコンマで区切ると、「\cref{eq:quadratic-equation,eq:with-align,eq:with-gather}」のように、自動的にそれらしい表記に変換してくれる (コンマとラベルの間にスペースは入れられない)。また連続した番号を物色を引用すると、「\cref{eq:example-line-1,eq:example-line-2,eq:with-align,eq:with-gather}」のように表記される。

数式を複数行で表示する時、かつては\texttt{eqnarray}を使うことが多かったが、最近は\texttt{align}あるいは\texttt{gather}を使うのが一般的。\texttt{align}の場合、\texttt{eqnarray}と同様、式を揃える時には\&を使う。
%
\begin{align}
  a &= a', & b &= b', & c &= c', \label{eq:example-line-1} \\
  a &= a', & b &= b', & c &= c'. \label{eq:example-line-2}
\end{align}
%
一方で\texttt{gather}は式を\&の位置で揃えるのではなく、各行を中央ぞろえにする。
%
\begin{titlebox}{\texttt{align}を使った場合}
  \adjustdisplayskips{1mm}
  \begin{align}
    \mathbf{x} = \argmin_{\mathbf{x}} \| \mathbf{Ax} - \mathbf{b} \|^2 \nonumber \\
    \mbox{where} \quad \mathbf{A} \in \mathbb{R}^{n \times m}, \quad \mathbf{b} \in \mathbb{R}^n \label{eq:with-align}
  \end{align}
\end{titlebox}
%
\begin{titlebox}{\texttt{gather}を使った場合}
  \adjustdisplayskips{1mm}
  \begin{gather}
    \mathbf{x} = \argmin_{\mathbf{x}} \| \mathbf{Ax} - \mathbf{b} \|^2 \nonumber \\
    \mbox{where} \quad \mathbf{A} \in \mathbb{R}^{n \times m}, \quad \mathbf{b} \in \mathbb{R}^n \label{eq:with-gather}
  \end{gather}
\end{titlebox}
%
上記の二つの式で\cref{eq:with-align}が\texttt{align}を使ったもの、\cref{eq:with-gather}が\texttt{gather}を使ったものである。

なお、ベクトルや行列の表記についてだが、計算機科学の分野では「Numerical Recipes in C」~\cite{numrecipes}という本の書き方に倣い、ベクトルを$\mathbf{x}$のようなアルファベット小文字のボールド体、行列を$\mathbf{A}$のようなアルファベット大文字のボールド体で書く。応用数学の分野などでは斜体かつボールドでない文字を使う文化もあるが、$\bm{x}$のようなイタリック・ボールド体はあまり使わないので注意。

\subsection{数式の例いろいろ}

\begin{titlebox}{積分}
  \adjustdisplayskips{1mm}
  \begin{equation}
    I = \int_0^{\infty} x^2 \mathrm{d}x
  \end{equation}
  \begin{center}
    ※ $dx$は正式には$\mathrm{d}x$のように$d$を斜体にせずに書く
  \end{center}
\end{titlebox}

\begin{titlebox}{和の計算}
  \adjustdisplayskips{1mm}
  \begin{equation}
    S = \sum_{i=0}^N x^2
  \end{equation}
\end{titlebox}

\begin{titlebox}{行列}
  \adjustdisplayskips{1mm}
  \begin{equation}
    \mathbf{A} = \begin{bmatrix}
      a_{0,0} & a_{0,1} & \cdots & a_{0,n-1} \\
      a_{1,0} & a_{1,1} & \cdots & a_{1,n-1} \\
      \vdots & \vdots & \ddots & \vdots \\
      a_{n-1,0} & a_{n-1,1} & \cdots & a_{n-1,n-1}
    \end{bmatrix}
  \end{equation}
\end{titlebox}

% ------------------------------------------------------------------------------
\section{数値や単位の書き方}
\label{sec:numbers-and-units}
% ------------------------------------------------------------------------------

数値や単位を書く場合は、通常通り数字を書けば問題ないが、\texttt{siunitx}パッケージを使うと便利な場合がある。例えば、有効数字を明らかにして$1.0 \times 10^3$のように書く場合、数式モードで書き下しても良いが、\texttt{siunitx}を用いると\texttt{\textbackslash num\{1.0e3\}}のように書くだけで「\num{1.0e3}」と出力される。またコンマで3桁ごとに区切られた数字も\texttt{siunitx}を用いることで\num{1000000} (\textbackslash num\{1000000\})のように書くことができる (\texttt{pethesis.sty}ファイルにて\texttt{\textbackslash sisetup\{group-separator=\{,\}\}}を指定している)。

単位を付ける場合には、チルダ記号「$\sim$」を使って数字と単位の間に半角スペースを入れ、10~kVのように書く。このような単位付きの表記の場合も、同様に\texttt{siunitx}パッケージを使うと便利な場合がある。同じ出力は\texttt{\textbackslash qty\{10\}\{\textbackslash kV\}}のように書くと得られ、出力は「\qty{10}{\kV}」のようになる。特に$\mathrm{\mu m}$のようにギリシャ文字が含まれるような単位は\texttt{\textbackslash qty\{10\}\{\textbackslash um\}} (出力: \qty{10}{\um})のように書く方がフォントを揃えるのが容易である。この他にも様々な単位が事前定義されているので、詳しくはドキュメントを確認すると良い。なお、角度の記号などはスペースを入れずに書くのが通例で、\texttt{siunitx}では\texttt{\textbackslash ang\{90.0\}}のように書くことで、「\ang{90.0}」という出力が得られる。

% ------------------------------------------------------------------------------
\section{図の入れ方}
\label{sec:insert-figure}
% ------------------------------------------------------------------------------

一昔前は日本語を\latex で使うために「pLaTeX」+「dvipdfmx」を使う必要があり、一度中間のDVIファイルを作る都合から画像ファイルをEPS形式で要する必要があった。 一方、現在は\pdflatex のほか\lualatex や\xelatex など、DVIを介せずに直接PDFを出力できる\tex を使用するため、画像はPDFファイルとして用意するのが一般的である。JPEGファイルを直接取り込むことも可能であるが、できる限りPDFで用意することを推奨する。

画像の取り込みには\texttt{figure}環境と\texttt{includegraphics}命令を以下のように組み合わせる。結果は\cref{fig:remeshing}のようになる。なお、本テンプレートは画像のPDFファイルを\texttt{figs}フォルダに配置しているが\texttt{pethesis.tex}のプリアンブルに\texttt{graphicspath}を指定しているため、このフォルダの中に画像ファイルが入っている場合には、フォルダを別段指定する必要はない。

\begin{lstlisting}[language=tex,caption={図の貼り方}]
  \begin{figure}[tb]
    \centering
    \includegraphics[width=\linewidth]{triangulation}
    \caption{Stanford bunnyモデルのリメッシュ結果}
    \label{fig:remeshing}
  \end{figure}  
\end{lstlisting}

\begin{figure}[!h]
  \centering
  \includegraphics[width=\linewidth]{triangulation}
  \caption{Stanford bunnyモデルのリメッシュ結果}
  \label{fig:remeshing}
\end{figure}

\begin{figure}[tbp]
	\centering
	\begin{tabular}{ccc}
    \includegraphics[width=.3\linewidth]{gtekt1} &
    \includegraphics[width=.3\linewidth]{gtekt2} &
    \includegraphics[width=.3\linewidth]{gtekt3}
  \end{tabular}
	\caption[溶接工程における大型部品運搬の様子]{溶接工程における大型部品運搬の様子 (図は株式会社ジーテクトのウェブページ~\cite{GTekt}より引用)}
	\label{fig:welding}
\end{figure}

また、複数の図を\cref{fig:welding}のような形で、\texttt{figure}と\texttt{tabular}を使ってレイアウトすることもできるが、図のレイアウトの自由度が低く、図の間のスペースが広くなる傾向にあるので、個人的にはおすすめはしない。なるべく、一枚の図を作って取り込む方が見た目はきれいになる。

図の配置については、通常、図はページの上に揃えて配置することが一般的であるので\texttt{\textbackslash \{figure\}[t]}のような形でtopに揃えることを意味するtを添えることが多い。この他、図が下に揃ってもいい場合には「b」、図を1ページを丸々使って表示したい場合には「p」を指定する。その場に表示したい場合には「h」あるいは「!h」を指定すればよいが、通常あまり好ましくないとされている。

キャプションを書く際、特に図目次がある学位論文においては、キャプションが長いせいで図目次が見づらくなることがある。この場合は、\texttt{\textbackslash caption[図目次用のキャプション]\{図の下に出るキャプション\}}のように書くことで、図目次専用のキャプションを指定できる。

% -----------------------------------------------
\subsection{図を入れる際の注意事項}

\begin{figure}[tbp]
  \centering
  \includegraphics[width=\linewidth]{sinogram_polygonizer}
  \caption[Sinogram Polygonizer法の概念図]{Sinogram Polygonizer法の概念図 (Yamanaka et al.~\cite{yamanaka2013sinogram}, \copyright 2013 IEEE)。}
  \label{fig:sinogram-polygonizer}
\end{figure}

図を別の論文から直接貼り付けることは一般的には好ましくなく、学会や論文誌に投稿する論文においては、ほとんどなされることはない。ただし、書籍や学位論文においては、一定のルールの元で引用することは可能と考えられている。厳密には、画像の著作権を有する学会や個人に許可をとった上で、掲載料を払って掲載する。主要な学会ではIEEEなどは学位論文に関しては掲載料が無料だが、ACMはかなり高額な掲載料がかかるので、特に一般に閲覧可能となる博士論文などは正規の手続きが求められる。引用に際しては\cref{fig:sinogram-polygonizer}のように、図のキャプションに論文のコピーライトを掲載する。


% ------------------------------------------------------------------------------
\section{表の書き方}
\label{sec:insert-table}
% ------------------------------------------------------------------------------

表の書き方については詳しく触れないが、テーブルの間の罫線については\texttt{\textbackslash hline}ではなく、\texttt{booktabs}パッケージの\texttt{\textbackslash toprule} (テーブルの上端の線)、\texttt{\textbackslash midrule} (テーブルの中間に使う線)、\texttt{\textbackslash bottomrule}のように太さが異なる線を使う方が見やすい。特に近年のCG分野の研究では、縦線を使うことはほとんどなく、\texttt{\textbackslash cmidrule} を使ってテーブルの横線に切れ目を入れることで区別をする場合が多いように感じる。

加えてテーブルの横幅だが、最近は\cref{tab:num-fruits-tblr}のようにページの幅いっぱいにテーブルを広げているものを多く見る。テーブルを横方向一杯に広げる方法はいくつかあるが、最も簡単なのは\texttt{tabularray}パッケージを使う方法である。\texttt{tabularray}パッケージでは、「\texttt{\textbackslash begin{tblr}\{colspec=\{列フォーマット指定\}, width=\textbackslash linewidth\}}」のように書くことでテーブルの幅を指定できる他、列のフォーマット指定時に「\texttt{lcr}」などの代わりに\texttt{X[l]}, \texttt{X[c]}, \texttt{X[r]}のように書くことで、テーブルの幅を保ったまま、各列の文字揃えを指定できる。これ以外にも、\texttt{{X[2,l]X[1,c]X[1,c]}}のように指定することで、各列の幅の比 (この例では2:1:1)などを指定することもできる。ただし、\texttt{tabularray}パッケージは\latex 3でないと使えないので注意。

\begin{table}[tb]
  \centering
  \caption{\texttt{tblr}環境を使った場合。}
  \label{tab:num-fruits-tblr}
  \begin{tblr}{colspec={X[2,r]X[3,c]X[3,c]X[3,c]}, width=\linewidth}
    \toprule
    名前 & りんご & みかん & バナナ \\
    \cmidrule{1-1} \cmidrule[l]{2-4}  % tabularrayの場合は()が[]に変わるので注意
    個数 & 1個 & 2個 & 3個 \\
    \bottomrule
  \end{tblr}
\end{table}

\begin{table}[tb]
  \centering
  \caption{列の幅を個別に指定した場合。}
  \label{tab:num-fruits-tabular}
  \begin{tblr}{
    column{1}={0.15\linewidth,r},
    column{2,3,4}={0.22\linewidth,c}
  }
    \toprule
    名前~~ & りんご & みかん & バナナ \\
    \cmidrule[r]{1-1} \cmidrule[l]{2-4}
    個数~~ & 1個 & 2個 & 3個 \\
    \bottomrule
  \end{tblr}
\end{table}


% ------------------------------------------------------------------------------
\section{アルゴリズムの書き方}
\label{sec:write-algorithm}
% ------------------------------------------------------------------------------

アルゴリズムの書き方については、機能が多岐に渡るので、冒頭でも紹介したWikibookなどを参照すること。実例として挿入ソートのアルゴリズムを\cref{alg:example-algo}に示してある。
\begin{itemize}
  \item \textsf{LaTeX/Algorithms - Wikibook} \\(\url{https://en.wikibooks.org/wiki/LaTeX/Algorithms})
\end{itemize}

なお、何らかのアルゴリズムを説明する際に、以下のような\texttt{algorithm}環境を用いて擬似コードを記述することは必須ではない。あくまで疑似コードは「本文を補助するためのもの」であり、擬似コードを示しただけでは説明したことにはならない。逆に本文で明快な説明ができているのであれば、あえて擬似コードを掲載する必要はない。

\begin{algorithm}[!h]
  \caption{アルゴリズムの例 (挿入ソート)}
  \label{alg:example-algo}
  \begin{algorithmic}[1]
    \Require{$A$: A list of numbers.}
    \Ensure{$A'$: The sorted list of $A$.}
    \Procedure{Insertion-Sort}{$A$}
        \For{$j = 2 \, \ldots \, A.\mathrm{length}$}
        \State{$k = A[j]$}
            \State{$i = j - 1$}
            \While{$i > 0$ and $A[i] > k$}
                \State{$A[i + 1] = A[i]$}
                \State{$i = i - 1$}
            \EndWhile
            \State{$A[i + 1] = k$}
        \EndFor
    \EndProcedure
  \end{algorithmic}
\end{algorithm}

% ------------------------------------------------------------------------------
\section{文献の引用の仕方}
\label{sec:cite-articles}
% ------------------------------------------------------------------------------

文献の引用と言えば\bibtex を使うのが一般的だったが、最近は機能を拡張した\biblatex というパッケージがあり、この文章では後者を使用している。ほとんど使い方は変わらないが、文献リストをプロパティで細かく設定できる分、\biblatex のほうが自由度が高い。例えばbackrefと呼ばれる文献が何ページで使われているかという情報を文献リストに付け加えることができる。引用の仕方や\texttt{*.bib}ファイルの書き方に関しては、\bibtex と\biblatex で大きな違いはない。ただし、\biblatex を使うためには\bibtex とは異なるスタイル指定のファイルが必要になるので、学会等に論文を投稿する際、\biblatex 用のスタイルファイルが用意されていなければ使うことはできない (\bibtex 用のスタイルファイルはほとんどの場合、提供されている)。

文献を個別に引用する場合には\texttt{\textbackslash cite\{hoppe96progressive\}}のように書く~\cite{hoppe96progressive}。なお、通常引用部分と直前の文章の間にはチルダの記号「$\sim$」を使って半角スペースを入れる。二つ以上を一度に引用する場合には\texttt{\textbackslash cite}の中括弧の中にコンマ区切りで並べるだけでよい~\cite{garland97surface,kazhdan06poisson, hoppe96progressive,qi2017pointnetpp}。なお、この文書では\biblatex の設定で、適当な順で\texttt{cite}の中に並べても番号順にソートされるようになっている。

また、著者名が日本語の場合、そのままだと漢字の最初の文字だけが抜き出されてしまうため、\texttt{author=\{\{谷田川 達也 and 大竹 豊 and 鈴木 宏正\}\}}のように\texttt{*.bib}ファイル内の著者名の部分で波括弧を二重にして囲う必要がある。
